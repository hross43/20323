%%%%%%%%%%%%%%%%%%%%%%%%%%%%%%%%%%%%%%%%%%%%%%%%%%%%%%%%%%%%
%%%%%%%%%%%%%%%%%%%%%%%%%%%%%%%%%%%%%%%%%%%%%%%%%%%%%%%%%%%%
%%%%%%%%%%%%%%%%%%%%%%%%%%%%%%%%%%%%%%%%%%%%%%%%%%%%%%%%%%%%
%%%%%%%%%%%%%%%%%%%%%%%%%%%%%%%%%%%%%%%%%%%%%%%%%%%%%%%%%%%%
%%%%%%%%%%%%%%%%%%%%%%%%%%%%%%%%%%%%%%%%%%%%%%%%%%%%%%%%%%%%
\documentclass[12pt]{article}
\usepackage{epsfig}
\usepackage{times}
\usepackage{amsmath}
\renewcommand{\topfraction}{1.0}
\renewcommand{\bottomfraction}{1.0}
\renewcommand{\textfraction}{0.0}
\setlength {\textwidth}{6.6in}
\hoffset=-1.0in
\oddsidemargin=1.00in
\marginparsep=0.0in
\marginparwidth=0.0in
\setlength {\textheight}{9.0in}
\voffset=-1.00in
\topmargin=1.0in
\headheight=0.0in
\headsep=0.00in
\footskip=0.50in                                         
%\setcounter{page}{1}
\begin{document}
\def\pos{\medskip\quad}
\def\subpos{\smallskip \qquad}
\newfont{\nice}{cmr12 scaled 1250}
\newfont{\name}{cmr12 scaled 1080}
\newfont{\swell}{cmbx12 scaled 800}
%%%%%%%%%%%%%%%%%%%%%%%%%%%%%%%%%%%%%%%%%%%%%%%%%%%%%%%%%%%%
%     DO NOT CHANGE ANYTHING ABOVE THIS LINE
%%%%%%%%%%%%%%%%%%%%%%%%%%%%%%%%%%%%%%%%%%%%%%%%%%%%%%%%%%%%
%     DO NOT CHANGE ANYTHING ABOVE THIS LINE
%%%%%%%%%%%%%%%%%%%%%%%%%%%%%%%%%%%%%%%%%%%%%%%%%%%%%%%%%%%%
%     DO NOT CHANGE ANYTHING ABOVE THIS LINE
%%%%%%%%%%%%%%%%%%%%%%%%%%%%%%%%%%%%%%%%%%%%%%%%%%%%%%%%%%%%
    \begin{center}
        {\bf PHYS 20323/60323: Fall 2020 - LaTeX Example}
        \vskip0.25in
    \end{center}

    \begin{enumerate}
        \item Consider a particle confined in a two-dimensional infinite square well
        
        \begin{center}
            $
                V(x,y)= \begin{cases}
                    0, & 0\leq x\leq a,\; 0\le y\le a\\
                    \infty, & otherwise
                \end{cases}
            $
        \end{center}

        The eigenfunctions have the form:
        \begin{center} 
                $\Psi(x,y)= \frac{a}{2} \sin(\frac{n \pi x}{a}) \sin(\frac{m \pi y}{a})$
        \end{center} 

        with the corresponding energies being given by:
        \vskip0.1in
        \begin{center}
            $E_{nm}= (n^2 + m^2) \frac{\pi^2 \hbar^2}{2ma^2}$
        \end{center}

        \begin{enumerate}
            \item (5 points) What are the levels of degeneracy of the five lowest energy values?
            \item (5 points) Consider a perturbation given by: 
            \vskip0.1in
            \begin{center}
                $\hat{H}' = a^2 V_0 \delta(x-\frac{a}{2}) \delta(y-\frac{a}{2})$
            \end{center}
            Calculate the first order correction to the ground state energy.
        \end{enumerate}

        \item{\bf The following questions refer to stars in the Table below.}\\
        Note: There may be multiple answers.
        
        \begin{center}
            \begin{tabular}{| l | l | l| l | l | l |}\hline
                Name & Mass & Luminosity & Lifetime & Temperature & Radius \\\hline
                Zeta & 60. M_{Sun} & 10^6 L_{Sun} &  8.0 x 10^5 years & & \\\hline
                Epsilon & 6.0 M_{Sun}  &  10^3 L_{Sun} & & 20,000 K & \\\hline
                Delta & 2.0 M_{Sun} & & 5.0 x 10^8 years & & 2 R_{Sun} \\\hline
                Beta & 1.3 M_{Sun} & 3.5 L_{Sun} & & & \\\hline
                Alpha & 1.0 M_{Sun} & & & & 1 R_{Sun}\\\hline
                Gamma & 0.7 M_{Sun} & & 4.5 x 10^{10} years & 5000 K & \\\hline
            \end{tabular}
            \vskip 0.25in
        \end{center}

        \begin{enumerate}
            \item (4 points) Which of these stars will produce a planetary nebula at the end of their life.
            \vskip0.5in
            \item (4 points) Elements heavier than Carbon will be produced in which stars.
        \end{enumerate}

    \end{enumerate}

\end{document}


